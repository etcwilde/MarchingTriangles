%%% Marching Triangles.tex
%%% 

%%%
%%% - ``review'' for content submitted for review
%%% - ``preprint'' for accepted content making available.
%%% - ``tog'' for technical papers accepted to the TOG joirnal and
%%%	for presentation at the SIGGRAPH or SIGGRAPH Asia conference.
%%% - ``conference'' for final content accepted to a sponsored event
%%% 	(hint: if unsure, use ``conference'')

%%% \documentclass[review]{acmsiggraph}
\documentclass[conference]{acmsiggraph}

\def\BibTeX{{\rm B\kern-.05em{\sc i\kern-.025em b}\kern-.08em
    T\kern-.1667em\lower.7ex\hbox{E}\kern-.125emX}}

%%% Used by the review variation

\TOGonlineid{V00775033}

%%% Used by the ``preprint'' variation
\TOGvolume{0}
\TOGnumber{0}

\title{An implementation of Marching Triangles}
\author{
	Evan T. C. Wilde \\
	Univeristy of Victoria\\
	etcwilde@uvic.ca
}
\pdfauthor{Evan T. C. Wilde}
\keywords{iso-surface, polygonization, marching-triangles}

\begin{document}
%%% This is the ``teaser'' command, which puts an figure, centered, below 
%%% the title and author information, and above the body of the content.

%%% \teaser{
%%%   \includegraphics[height=1.5in]{images/sampleteaser}
%%%   \caption{Spring Training 2009, Peoria, AZ.}
%%% }

\maketitle

\begin{abstract}
	In this paper, we describe a method of implementing the marching
	triangles algorithm. The marching triangle algorithm begins with a
	seed triangle and grows the mesh over an implicit surface in an
	iterative process. This allows the algorithm to generate a mesh that
	more closely represents the topology of the surface while minimizing
	polygons.
\end{abstract}

\keywordlist

\copyrightspace

\section{Introduction}

\section{Related Work}

\section{Marching Triangles Algorithm}



\end{document}

