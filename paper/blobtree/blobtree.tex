% vim: set fecn=utf8 ft=latex encoding=utf8
% -*- mode: latex; coding: UTF-8; -*-

\newif\ifdraft
\drafttrue

%%% blobtree.tex

\ifdraft
	\documentclass[conference]{acmsiggraph}
	\def\baselinestrech{1}
	\setlength{\marginparwidth}{2cm}
	\newcommand{\Title}{Implementing the Blob Tree | Draft}
\else
	\documentclass[conference]{acmsiggraph}
	\newcommand{\Title}{Implementing the Blob Tree}
\fi

\newcommand{\Author}{Evan T. C. Wilde}
\newcommand{\Email}{etcwilde@uvic.ca}
\newcommand{\Subject}{Implementing the Blob Tree}
\newcommand{\Keywords}{implicit surface, iso-surface, blobs, blobbies, blobtree}

\synctex=1

\usepackage{xspace}
\usepackage{amssymb}
\usepackage{amsmath}
\usepackage{svg}

\hypersetup{pdftitle={\Title},
        pdfauthor={\Author},
        pdfkeywords={\Keywords},
        pdfsubject={\Subject},
        urlcolor=blue,citecolor=black}

\def\BibTeX{{\rm B\kern-.05em{\sc i\kern-.025em b}\kern-.08em
    T\kern-.1667em\lower.7ex\hbox{E}\kern-.125emX}}

\TOGonlineid{000000000}
\TOGvolume{0}
\TOGnumber{0}

\ifdraft
	\usepackage[colorinlistoftodos]{todonotes}
		\newcommand{\evan}[1]{{\color{cyan}\emph{Evan Says:
		#1}}\xspace}
		\newcommand{\evanTodo}[1]{{\color{blue}\emph{Evan Todo:
		#1}}\xspace}
\else
	\usepackage[disable]{todonotes}
	\newcommand{\evan}[1]{}
	\newcommand{\evanTodo}[1]{}
\fi


%%% Local Variables:
%%% mode: latex
%%% TeX-master: "marching-triangles"
%%% End:


\title{\Title}
\author{
	\Author \\
	University of Victoria\\
\Email}
\pdfauthor{\Author}
\keywords{\Keywords}

\begin{document}

\teaser{
	\includegraphics[height=1.5in] {images/blend.png}
	\caption{Blended Blobs}
}

\maketitle

\begin{abstract}
	Implicit surfaces are a mathematical representation of geometric
	information; storing complex geometric information with minimal memory
	requirements. Blobs are a form of implicit object defined by a central
	origin point, a \fff, and an iso value. The surface is defined at
	radius $r$ when the \fff evaluated on $r$ is equal to the iso value.

	A blob tree is used to build complex geometry from simpler
	primitives. Intelligent construction of the blob-tree is critical for
	fast evaluation of the implicit object represented in the tree, whether
	that be for ray-tracing or polygonization.

\end{abstract}

\keywordlist

\copyrightspace

\section{Introduction}
Blob trees are a data-structure for containing and evaluating the geometric
information stored by blobs.
\section{Related Work}
\section{Implementation Details}
\subsection{Finding Surface}
Finding the point where the surface exists proves to be challenging, as the
\fff\ is not necessarily invertible. To find the surface, a numerical
root-finding method is required. I used the following secant method.
$$x_i = x_{i-1} - f(x_{i-1}) \times {x_{i-1} - x_{i-2} \over f(x_{i-1}) -
f(x_{i-2})}$$

The secant method was chosen due to the relative speed of convergence while not
requiring derivations. The bisection method guarantees convergence and does not
require differentiation; however, it requires a bracket to be placed on the
surface, and has linear convergence. The Newton-Raphson method is an open
method that has order $n^2$ convergence, but requires derivations and cannot
guarantee convergence. The secant method is another open method that has order
$n^1.618$ convergence and does not require derivations, but cannot guarantee
convergence. Furthermore, as an open interval method, the initial point must be
close enough to the shape that the \fff\ has an effect.
\subsection{Transforms}
\subsection{Operations}
\subsubsection{Blend}
\subsubsection{Union}
\subsubsection{Intersect}
\subsection{Instancing}
\subsection{Normal Data}
Normal is defined by the normalized weighted linear combination of the normal
vectors of the contributing component spheres, where the weights are defined by
the contribution to the iso value of the corresponding component
sphere\cite{Wyvill}.
\subsection{Curvature}
We are looking for the principal curvatures at a given point on the surface.
The principal curvatures $k1$ and $k2$, define the curvature along the tangent
and bi-normal vectors to the given point.
Given a point $p \in \mathbb{R}^3$. Magic with matrices
happens\cite{DeAraujo2004}.
\subsection{Bounding Volume Hierarchies}
Axis-aligned bounding-boxes (AABB).

\bibliographystyle{acmsiggraph}
\bibliography{references}
\end{document}

