% vim: set fecn=utf8 ft=latex encoding=utf8
% -*- mode: latex; coding: UTF-8; -*-

\newif\ifdraft
\drafttrue

%%% blobtree.tex

\ifdraft
	\documentclass[conference]{acmsiggraph}
	\def\baselinestrech{1}
	\setlength{\marginparwidth}{2cm}
	\newcommand{\Title}{Implementing the Blob Tree | Draft}
\else
	\documentclass[conference]{acmsiggraph}
	\newcommand{\Title}{Implementing the Blob Tree}
\fi

\newcommand{\Author}{Evan T. C. Wilde}
\newcommand{\Email}{etcwilde@uvic.ca}
\newcommand{\Subject}{Implementing the Blob Tree}
\newcommand{\Keywords}{implicit surface, iso-surface, blobs, blobbies, blobtree}

\synctex=1

\usepackage{xspace}
\usepackage{amssymb}
\usepackage{amsmath}
\usepackage{svg}

\hypersetup{pdftitle={\Title},
        pdfauthor={\Author},
        pdfkeywords={\Keywords},
        pdfsubject={\Subject},
        urlcolor=blue,citecolor=black}

\def\BibTeX{{\rm B\kern-.05em{\sc i\kern-.025em b}\kern-.08em
    T\kern-.1667em\lower.7ex\hbox{E}\kern-.125emX}}

\TOGonlineid{000000000}
\TOGvolume{0}
\TOGnumber{0}

\ifdraft
	\usepackage[colorinlistoftodos]{todonotes}
		\newcommand{\evan}[1]{{\color{cyan}\emph{Evan Says:
		#1}}\xspace}
		\newcommand{\evanTodo}[1]{{\color{blue}\emph{Evan Todo:
		#1}}\xspace}
\else
	\usepackage[disable]{todonotes}
	\newcommand{\evan}[1]{}
	\newcommand{\evanTodo}[1]{}
\fi


%%% Local Variables:
%%% mode: latex
%%% TeX-master: "marching-triangles"
%%% End:


\title{\Title}
\author{
	\Author \\
	University of Victoria\\
\Email}
\pdfauthor{\Author}
\keywords{\Keywords}

\begin{document}

\teaser{
	\includegraphics[height=1.5in] {images/blend.png}
	\caption{Blended Blobs}
}

\maketitle

\begin{abstract}
	Implicit surfaces are a mathematical representation of geometric
	information; storing complex geometric information with minimal memory
	requirements. Blobs are a form of implicit object defined by a central
	origin point, a \fff, and an iso value. The surface is defined at
	radius $r$ when the \fff evaluated on $r$ is equal to the iso value.

	A blob tree is used to build complex geometry from simpler
	primitives. Intelligent construction of the blob-tree is critical for
	fast evaluation of the implicit object represented in the tree, whether
	that be for ray-tracing or polygonization.

\end{abstract}

\keywordlist

\copyrightspace

\section{Introduction}
Blob trees are a data-structure for containing and evaluating the geometric
information stored by blobs. A blob tree represents an object stored as
primitive objects stored in a hierarchy to build more complex geometries. The
nodes represent the various transformations, binary operations, and primitive
shapes that are combined in the hierarchy, allowing the design of more complex
objects.

In this paper, we explore the implementation details of a simple blob tree
structure and various additions to increase the performance and robustness of
the tree.

The implementation of the implicit system is available on my
github\footnote{\url{https://github.com/etcwilde/ImplicitSystem}}, and the
implementation of the raytracer is also
available\footnote{\url{https://github.com/etcwilde/ImplictRaytacer}}.

\section{Related Work}

\section{Implementation Details}
In this section, we describe the methods of implementing the blob tree, and why
we made a given decision.
\subsection{Finding Surface}
Finding the point where the surface exists proves to be challenging, as the
\fff\ is not necessarily invertible. To find the surface, a numerical
root-finding method is required. I used the secant method, described as
follows:
$$x_i = x_{i-1} - f(x_{i-1}) \times {x_{i-1} - x_{i-2} \over f(x_{i-1}) -
f(x_{i-2})}$$

The secant method was chosen due to the relative speed of convergence, not
requiring bracketing, and not requiring derivations. The bisection method
guarantees convergence and does not require differentiation; however, it
requires a bracket to be placed on the surface, and has linear convergence. The
Newton-Raphson method is an open method that has order $n^2$ convergence, but
requires derivations and cannot guarantee convergence. The secant method is
another open method that has order $n^{1.618}$ convergence and does not require
derivations, but cannot guarantee convergence. Furthermore, as an open interval
method, the initial point must be close enough to the shape that the \fff\ has
an effect.

For the purpose of polygonization, it is possible to use the centre point of a
given primitive, which will be at the centre of the field function. In most
cases, given the Geoff \fff, the secant method will converge within 6
iterations. For this purpose, no extra measures are necessary to ensure
convergence.

\begin{figure}[htb]
	\centering
	\includegraphics[height=1.5in] {images/sphere_no_bvh.png}
	\caption{Ray-traced blob without bvh}
	\includegraphics[height=1.5in] {images/sphere_bvh.png}
	\caption{Ray-traced blob with bvh}
\end{figure}

For the purpose of raytracing, in most cases the rays being sent toward the
object are beyond the maximum radius of the \fff\ and the secant method is
unable to converge. For this, we implemented a simple bounding volume
hierarchy, which is able to quickly determine which rays get close enough to
the surface that the \fff\ will have values different than zero.

\subsection{Transforms}
We implemented the three basic transforms: translation, rotation, and scaling.
The transform class simply takes any 3-dimensional homogeneous matrix, making
the implementation of the translation, rotation, and scaling operations
trivial. The use of an over-arching transform class simplifies.

\begin{figure}[htb]
	\centering
	\includegraphics[height=1.5in]{images/intersect_rotate.png}
	\caption{Intersected Spheres Rotated then Translated}
\end{figure}

\subsection{Operations}
Blob-trees are able to perform various binary operations between two
sub-trees. Our implementation provides three of the five possible operations;
the blend, union, and intersect operations. The two remaining operations are
the warp, which is not a binary operation, but includes tapers and twisting,
and difference, allowing blobs to remove volume from another blob.

We describe the method of calculating the field value and normal vector for
each given operation.

The method for calculating the field values of the various operations is
described thoroughly in \textit{The BlobTree} \cite{Wyvill}.

\begin{enumerate}
	\item Primitive:\\
		$F(p)$\\
		$N(p) = {center - p \over | center - p |}$
	\item Blend:\\
		$F(L(o), p) + F(R(o), p)$\\
		$N( $

	\item Union:\\
		$max(F(L(o), p), F(R(o), p))$
	\item Intersect:\\
		$min(F(L(o), p), F(R(o), p))$
\end{enumerate}

The primitive does not contain any children, and therefore will simply
evaluate the field value with respect to the distance from the centre point.
Similarly for calculating the normal vector, the primitive has no children, so
the normal will be calculated as follows.

\subsection{Instancing}
By using instancing, we are able to minimize the memory usage by not
duplicating simpler geometry to build more complex geometry. As an example, a
union object can be made of a single primitive $p$, translated by some
distance. We get the effect of two separate primitive instances, but we only
use the memory of a single instance.

In implicit systems, instancing has further performance benefits as it
introduces the concept of local and world coordinates. An example of where this
is particularly useful is when scaling the sphere. It is a lot simpler to
perform a ray-sphere intersection than to perform the intersection test on a
ellipsoid. By mapping the ray into the local coordinate system, we are able to
leave the object as a sphere by bending the world and rays around it.
\subsection{Normal Data}
Normal is defined by the normalized weighted linear combination of the normal
vectors of the contributing component spheres, where the weights are defined by
the contribution to the iso-value of the corresponding component
sphere\cite{Wyvill}.
\subsection{Curvature}
We are looking for the principal curvatures at a given point on the surface.
The principal curvatures $k1$ and $k2$, define the curvature along the tangent
and bi-normal vectors to the given point. The method for finding the curvature
is described in section four of \textit{Curvature Formulas for Implicit
Surfaces} \cite{Goldman2005}. By using a cofactor approach, we can find the adjoint matrix to the
hessian, $H'$. 

$$
H'(F) =
\begin{bmatrix}
	F_{yy}F_{zz} - F_{yz}F_{zy} & F_{yz}F_{zx} - F_{yx}F_{zz} & F_{yx}F_{zy} - F_{yy}F_{zx} \\
	F_{xz}F_{zy} - F_{xy}F_{zz} & F_{xx}F_{zz} - F_{xz}F_{zx} & F_{xy}F_{zx} - F_{xx}F_{zy} \\
	F_{xy}F_{yz} - F_{xz}F_{yy} & F_{yx}F_{xz} - F_{xx}F_{yz} & F_{xx}F_{yy} - F_{xy}F_{yx}
\end{bmatrix}
$$
By performing the adjoint operation on $H'$, we can get the
hessian matrix $H$. Then we can use the mean $k_M$ and gaussian $k_G$
curvatures to find
the principal curvatures $k1$ and $k2$ with the formula:
$$k1, k2 = k_M \pm \sqrt{k^2_{M} - k_G}$$


$$k_M = {\nabla F * H'(F) * \nabla F^T \over |\nabla F|^4}$$

$$k_G = {\nabla F * H(F) * \nabla F^T - |\nabla F|^2 * H_{diagonal} \over 2 *
|\nabla F|^3}$$

I would like to investigate a method for finding curvature described in another
paper\cite{DeAraujo2004} to see which handles sharp-edges and discontinuities
better, and which method can be evaluated more quickly.

The method described by De Ara\'{u}jo uses the partial derivatives of the
normal vector in a single matrix, rather than computing multiple matrices.

$$
C=
\begin{bmatrix}
	{\partial N_x \over \partial x}&{\partial N_x \over \partial y}&{\partial N_x \over \partial z}\\
	{\partial N_y \over \partial x}&{\partial N_y \over \partial y}&{\partial N_y \over \partial z}\\
	{\partial N_z \over \partial x}&{\partial N_z \over \partial y}&{\partial N_z \over \partial z}
\end{bmatrix}
$$

The matrix C can be defined using the hessian matrix and the normal vector.

\subsection{Bounding Volume Hierarchies}
We implemented the bounding volume hierarchy (bvh) using axis-aligned bounding
boxes(AABB). We used this construction due to the simplicity and speed of the
intersection testing. A paper XXX describes an optimized method of determining
the point of intersection of a ray with the AABB.

Axis-aligned bounding-boxes (AABB).

\section{Findings}
In this section, we describe how the decisions made effected the resulting
speed and robustness of the blob tree.

\subsection{Bounding Volume Hierarchies}
We found that while the bvh greatly improved the rate of convergence of the
secant method, taking an average of four iterations with the bvh versus the
twenty iterations without, and increased the number of locations where it did
converge, we did not find that our naive approach had any effect on the overall
speed of the ray tracing program, which should have received the most benefit
from such a structure.


\bibliographystyle{acmsiggraph}
\bibliography{references}
\end{document}

